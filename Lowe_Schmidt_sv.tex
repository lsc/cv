\documentclass[10pt]{article}

% This is a helpful package that puts math inside length specifications
\usepackage{calc}
\usepackage[latin1]{inputenc}
\usepackage[swedish]{babel}
%%\usepackage[T1]{fontenc}
\usepackage{tgpagella}

% Simpler bibsection for CV sections
% (thanks to natbib for inspiration)
\makeatletter
\newlength{\bibhang}
\setlength{\bibhang}{1em}
\newlength{\bibsep}
 {\@listi \global\bibsep\itemsep \global\advance\bibsep by\parsep}
\newenvironment{bibsection}
    {\minipage[t]{\linewidth}\list{}{%
        \setlength{\leftmargin}{\bibhang}%
        \setlength{\itemindent}{-\leftmargin}%
        \setlength{\itemsep}{\bibsep}%
        \setlength{\parsep}{\z@}%
        }}
    {\endlist\endminipage}
\makeatother

% Layout: Puts the section titles on left side of page
\reversemarginpar

%
%         PAPER SIZE, PAGE NUMBER, AND DOCUMENT LAYOUT NOTES:
%
% The next \usepackage line changes the layout for CV style section
% headings as marginal notes. It also sets up the paper size as either
% letter or A4. By default, letter was used. If A4 paper is desired,
% comment out the letterpaper lines and uncomment the a4paper lines.
%
% As you can see, the margin widths and section title widths can be
% easily adjusted.
%
% ALSO: Notice that the includefoot option can be commented OUT in order
% to put the PAGE NUMBER *IN* the bottom margin. This will make the
% effective text area larger.
%
% IF YOU WISH TO REMOVE THE ``of LASTPAGE'' next to each page number,
% see the note about the +LP and -LP lines below. Comment out the +LP
% and uncomment the -LP.
%
% IF YOU WISH TO REMOVE PAGE NUMBERS, be sure that the includefoot line
% is uncommented and ALSO uncomment the \pagestyle{empty} a few lines
% below.
%

%% Use these lines for letter-sized paper
%\usepackage[paper=letterpaper,
%            %includefoot, % Uncomment to put page number above margin
%            marginparwidth=1.2in,     % Length of section titles
%            marginparsep=.05in,       % Space between titles and text
%            margin=1in,               % 1 inch margins
%            includemp]{geometry}

% Use these lines for A4-sized paper
\usepackage[paper=a4paper,
            %includefoot, % Uncomment to put page number above margin
            marginparwidth=30.5mm,    % Length of section titles
            marginparsep=1.5mm,       % Space between titles and text
            margin=25mm,              % 25mm margins
            includemp]{geometry}

%% More layout: Get rid of indenting throughout entire document
\setlength{\parindent}{0in}

%% This gives us fun enumeration environments. compactitem will be nice.
\usepackage{paralist}

%% Reference the last page in the page number
%
% NOTE: comment the +LP line and uncomment the -LP line to have page
%       numbers without the ``of ##'' last page reference)
%
% NOTE: uncomment the \pagestyle{empty} line to get rid of all page
%       numbers (make sure includefoot is commented out above)
%
\usepackage{fancyhdr,lastpage}
\pagestyle{fancy}
%\pagestyle{empty}      % Uncomment this to get rid of page numbers
\fancyhf{}\renewcommand{\headrulewidth}{0pt}
\fancyfootoffset{\marginparsep+\marginparwidth}
\newlength{\footpageshift}
\setlength{\footpageshift}
          {0.5\textwidth+0.5\marginparsep+0.5\marginparwidth-2in}
\lfoot{\hspace{\footpageshift}%
       \parbox{4in}{\, \hfill %
                    \arabic{page} of \protect\pageref*{LastPage} % +LP
%                    \arabic{page}                               % -LP
                    \hfill \,}}

% Finally, give us PDF bookmarks
\usepackage{color,hyperref}
\definecolor{darkblue}{rgb}{0.0,0.0,0.3}
\hypersetup{colorlinks,breaklinks,
            linkcolor=darkblue,urlcolor=darkblue,
            anchorcolor=darkblue,citecolor=darkblue}

%%%%%%%%%%%%%%%%%%%%%%%% End Document Setup %%%%%%%%%%%%%%%%%%%%%%%%%%%%


%%%%%%%%%%%%%%%%%%%%%%%%%%% Helper Commands %%%%%%%%%%%%%%%%%%%%%%%%%%%%

% The title (name) with a horizontal rule under it
%
% Usage: \makeheading{name}
%
% Place at top of document. It should be the first thing.
\newcommand{\makeheading}[1]%
        {\hspace*{-\marginparsep minus \marginparwidth}%
         \begin{minipage}[t]{\textwidth+\marginparwidth+\marginparsep}%
                {\large \bfseries #1}\\[-0.15\baselineskip]%
                 \rule{\columnwidth}{1pt}%
         \end{minipage}}

% The section headings
%
% Usage: \section{section name}
%
% Follow this section IMMEDIATELY with the first line of the section
% text. Do not put whitespace in between. That is, do this:
%
%       \section{My Information}
%       Here is my information.
%
% and NOT this:
%
%       \section{My Information}
%
%       Here is my information.
%
% Otherwise the top of the section header will not line up with the top
% of the section. Of course, using a single comment character (%) on
% empty lines allows for the function of the first example with the
% readability of the second example.
\renewcommand{\section}[2]%
        {\pagebreak[2]\vspace{1.3\baselineskip}%
         \phantomsection\addcontentsline{toc}{section}{#1}%
         \hspace{0in}%
         \marginpar{
         \raggedright \scshape #1}#2}

% An itemize-style list with lots of space between items
\newenvironment{outerlist}[1][\enskip\textbullet]%
        {\begin{itemize}[#1]}{\end{itemize}%
         \vspace{-.6\baselineskip}}


% An environment IDENTICAL to outerlist that has better pre-list spacing
% when used as the first thing in a \section
\newenvironment{lonelist}[1][\enskip\textbullet]%
        {\vspace{-\baselineskip}\begin{list}{#1}{%
        \setlength{\partopsep}{0pt}%
        \setlength{\topsep}{0pt}}}
        {\end{list}\vspace{-.6\baselineskip}}

% An itemize-style list with little space between items
\newenvironment{innerlist}[1][\enskip\textbullet]%
        {\begin{compactitem}[#1]}{\end{compactitem}}

% An environment IDENTICAL to innerlist that has better pre-list spacing
% when used as the first thing in a \section
\newenvironment{loneinnerlist}[1][\enskip\textbullet]%
        {\vspace{-\baselineskip}\begin{compactitem}[#1]}
        {\end{compactitem}\vspace{-.6\baselineskip}}

% To add some paragraph space between lines.
% This also tells LaTeX to preferably break a page on one of these gaps
% if there is a needed pagebreak nearby.
\newcommand{\blankline}{\quad\pagebreak[2]}

% Uses hyperref to link DOI
\newcommand\doilink[1]{\href{http://dx.doi.org/#1}{#1}}
\newcommand\doi[1]{doi:\doilink{#1}}


%%%%%%%%%%%%%%%%%%%%%%%% End Helper Commands %%%%%%%%%%%%%%%%%%%%%%%%%%%

%%%%%%%%%%%%%%%%%%%%%%%%% Begin CV Document %%%%%%%%%%%%%%%%%%%%%%%%%%%%

\begin{document}
\makeheading{Lowe Schmidt}

\section{Kontakt information}
%
\newlength{\rcollength}\setlength{\rcollength}{1.85in}%
%
\begin{tabular}[t]{@{}p{\textwidth-\rcollength}p{\rcollength}}
Jarl Kulles Gata 6				& \textit{Mobil:} +46 723 867 157\\
126 54 Hägersten			& \textit{E-mail:}
\href{mailto:jobs@loweschmidt.se} {jobs@loweschmidt.se}\\
\end{tabular}

\section{Erfarenhet}
%
\textbf{Konsult}, \href{http://www.init.se}{\textbf{Init}}\hfill {Sen November 2013}
\begin{outerlist}
	\item[]
\end{outerlist}
\blankline

\textbf{Systemadministratör}, \href{http://www.avanza.se}{\textbf{Avanza}}\hfill {April 2013 - Oktober 2013}
\begin{outerlist}
	\item[] Plattformsgruppen vid Avanza ansvarar för web- och handelsplattformarna på Avanza, mitt fokus rör huvudsakligen webbplatformen som är baserad på RHEL och Solaris, Oracles JVM och Gigaspaces för applikationslagret. Mitt intresse för automatisering och orkestrering gör att jag också ser över vår Puppet och Mcollective-installation, planerar uppgradering, utbildar kollegor i kodstandarder och idiom samt faktoriserar om.
\end{outerlist}
\blankline

\textbf{Systemadministratör}, \href{http://www.ztorm.com}{\textbf{Ztorm}}\hfill {Januari 2013 - Mars 2013
\begin{outerlist}
	\item[] Systemadministratör för en LAMP-stack baserad på Gentoo samt stödsystem baserade på bl.a OpenSolaris, FreeBSD och OpenBSD. Jag utförde även en hel del prestanda- och lasttester.
\end{outerlist}
\blankline

\textbf{Systemadministratör}, {\textbf{The Delta Projects}}\hfill {Augusti 2012 till December 2012}
\begin{outerlist}
	\item[] Tekniker för hela stacken, nätverksarkitektur, racka servrar i vårt nya datacenter och hela vägen upp till automatiserade OS-installationer med hjälp av The Foreman. För infastrukturautomatisering använde vi Chef.
\end{outerlist}
\blankline

\textbf{Infrastrukturtekniker}, \href{http://www.unibet.com}{\textbf{Unibet}}\hfill {Mars 2011 till Juli 2012}
\begin{outerlist}
	\item[] Som en del av infrastrukturteamet på Technical Development jobbade jag bland annat med lastbalansering, SAN och underhållet av vår ca 1000 maskiner stora maskinpark. Jag jobbade även en hel del med våra rutiner och processer i hur vi interagerar med andra team på Unibet. 
	
	\item Designade och implementerade vår ersättare till Red Hat Satellite, valet föll på  			\href{http://www.theforeman.org} {The Foreman} i kombination med \href{http://www.puppetlabs.com/puppet} {Puppet}. Planering av releaserutiner, versionshantering och skrev en stor del av vår tidiga Puppetkod.
	\item Föreslog och drev idén om att ha en 'målvakt' för alla frågor rörande driften i ett roterande schema, en person från varje team (Nätverk, Infrastruktur, Applikation) hade i uppgift att vara tillgänglig för frågor från utvecklare, projektledare och från våra systemadministratörer.
	\item Introducerade Kanban som arbetsflöde för Technical Development, satt även som 'Kanbanmaster' för prioriteringar och nedbrytande av stora uppgifter i mindre deluppgifter.
	
\end{outerlist}
\blankline

\textbf{Systemspecialist}, \href{http://www.24solutions.se}{\textbf{24 Solutions}} \hfill {September 2010 till Mars 2011}
\begin{outerlist}
	\item[] Vid 24Solutions jobbade jag huvudsakligen som konsult på bwin.games, där satt jag i deras support services team och jobbade med övervakning och grafning med Nagios och Cacti. Jag deltog också i deras förstudie om att använda Puppet och introducerade Kanban som ett arbetsflöde för Support Services.

\end{outerlist}
\blankline

\textbf{Systemadministratör}, {\textbf{Squace}} \hfill {Mars 2010 till Juli 2010}
\begin{outerlist}
	\item[] Systemadministration för ett litet startup, gjorde allt ifrån desktopsupport till infrastrukturdesign i samarbete med mjukvaruarkitekten. Här startade med intresse för automatisering och configuration management-verktyg. Jobbade med CentOS, Tomcat, PostgreSQL, VirtualBox och XenServer. I augusti 2010 lade Squace ner sin verksamhet i Sverige av ekonomiska skäl.
\end{outerlist}
\blankline

\textbf{Systemadministratör}, \href{http://www.r2m.se}{\textbf{R2M}}, \hfill {December 2007 till Augusti 2009}
\begin{outerlist} 
	\item[] Jobbade både internt som  systemadministratör men även ute hos kund, konsultade bland annat på SVT med drift WebLogic Integration och på Ledarna med IBMs WebSphere Express. Planerade och utförde migrationen av en föråldrard Windows 2000 AD till en baserad på Debian, OpenLDAP och Samba.
\end{outerlist}
\blankline

\section{Teknik}
GNU/Linux, TCP/IP, DNS (BIND, DNSmasq), SMTP (Postfix, Qmail), VRRP (Keepalived), Webservrar (Apache, Nginx), Relationsdatabaser (MySQL, SQLite), Lastbalanserare (BigIP F5), Puppet, MCollective, Perl, Ruby, Bash, Nagios, Graphite, Subversion, Git mm

\section{Utbildning}
%
{KY} {Linux och UNIX Systemadministration}, \href{http://www.jenseneducation.se}{\textbf{JENSEN Education}}\hfill {2005 till 2007}
\begin{outerlist}
\item[] Yrkesutbildning inom systemadministration med inriktning på Linux, kurser i bland annat TCP/IP, datorsäkerhet, databas- och webserveradministration samt skriptning för UNIX. En tredjedel av utbildningen var lärande i arbete, jag spenderade min tid på Avalanche Studios och AlaTEST
\end{outerlist}
\blankline

\section{Övriga engagemang}
{Grundare, Organisatör}, \href{http://www.meetup.com/DevOps-Stockholm/}{\textbf{Stockholm DevOps}} \hfill{Sen Mars 2011}
\begin{outerlist}
\item[] DevOps (sammanslagning av Developers och Operations) är ett sätt att utveckla mjukvara
		med betoning på kommunikation, samarbete och integration mellan alla delar av leveransorganisationen. Jag startade en lokal användargrupp för Stockholm och har både organiserat och föreläst vid flertalet tillfällen. \end{outerlist}
\blankline

{Organisatör, Webansvarig}, \href{http://bodyfest.se} {\textbf{Bodyfest}} \hfill{Sen Januari 2010}
\begin{outerlist}
	\item[] Bodyfest är ett årligt, större lievmusikarrangemang med betoning på elektronisk musik, ett litet axplock av band som stått på vår scen: Front 242, Pouppée Fabrikk, The Invincible Spirit, The Klinik m.m
\end{outerlist}
\blankline

{Organisatör, DJ}, {\textbf{Klubb Bodytåget}}, \hfill{Sen Augusti 2008}
\begin{outerlist}
	\item[] Bodytåget är en livemusikklubb för genrer som EBM, Synthpop och  Industrial.
			Jag är ekonomiskt ansvarig, organiserar och spelar skivor.
\end{outerlist}


\section{Referenser}
På begäran.
	
\section{Språk}
Flytande svenska, engelska och tyska i tal och skrift.


\end{document}

%%%%%%%%%%%%%%%%%%%%%%%%%% End CV Document %%%%%%%%%%%%%%%%%%%%%%%%%%%%%
