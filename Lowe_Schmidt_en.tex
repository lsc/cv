\documentclass[8pt]{article}

% This is a helpful package that puts math inside length specifications
\usepackage{calc}
%\usepackage[latin1]{inputenc}
\usepackage[swedish]{babel}
\usepackage[T1]{fontenc}
\usepackage{tgpagella}
\usepackage{eurosym}


% Simpler bibsection for CV sections
% (thanks to natbib for inspiration)
\makeatletter
\newlength{\bibhang}
\setlength{\bibhang}{1em}
\newlength{\bibsep}
 {\@listi \global\bibsep\itemsep \global\advance\bibsep by\parsep}
\newenvironment{bibsection}
    {\minipage[t]{\linewidth}\list{}{%
        \setlength{\leftmargin}{\bibhang}%
        \setlength{\itemindent}{-\leftmargin}%
        \setlength{\itemsep}{\bibsep}%
        \setlength{\parsep}{\z@}%
        }}
    {\endlist\endminipage}
\makeatother

% Layout: Puts the section titles on left side of page
\reversemarginpar

%
%         PAPER SIZE, PAGE NUMBER, AND DOCUMENT LAYOUT NOTES:
%
% The next \usepackage line changes the layout for CV style section
% headings as marginal notes. It also sets up the paper size as either
% letter or A4. By default, letter was used. If A4 paper is desired,
% comment out the letterpaper lines and uncomment the a4paper lines.
%
% As you can see, the margin widths and section title widths can be
% easily adjusted.
%
% ALSO: Notice that the includefoot option can be commented OUT in order
% to put the PAGE NUMBER *IN* the bottom margin. This will make the
% effective text area larger.
%
% IF YOU WISH TO REMOVE THE ``of LASTPAGE'' next to each page number,
% see the note about the +LP and -LP lines below. Comment out the +LP
% and uncomment the -LP.
%
% IF YOU WISH TO REMOVE PAGE NUMBERS, be sure that the includefoot line
% is uncommented and ALSO uncomment the \pagestyle{empty} a few lines
% below.
%

%% Use these lines for letter-sized paper
%\usepackage[paper=letterpaper,
%            %includefoot, % Uncomment to put page number above margin
%            marginparwidth=1.2in,     % Length of section titles
%            marginparsep=.05in,       % Space between titles and text
%            margin=1in,               % 1 inch margins
%            includemp]{geometry}

% Use these lines for A4-sized paper
\usepackage[paper=a4paper,
            %includefoot, % Uncomment to put page number above margin
            marginparwidth=30.5mm,    % Length of section titles
            marginparsep=1.5mm,       % Space between titles and text
            margin=25mm,              % 25mm margins
            includemp]{geometry}

%% More layout: Get rid of indenting throughout entire document
\setlength{\parindent}{0in}

%% This gives us fun enumeration environments. compactitem will be nice.
\usepackage{paralist}

%% Reference the last page in the page number
%
% NOTE: comment the +LP line and uncomment the -LP line to have page
%       numbers without the ``of ##'' last page reference)
%
% NOTE: uncomment the \pagestyle{empty} line to get rid of all page
%       numbers (make sure includefoot is commented out above)
%
\usepackage{fancyhdr,lastpage}
\pagestyle{fancy}
%\pagestyle{empty}      % Uncomment this to get rid of page numbers
\fancyhf{}\renewcommand{\headrulewidth}{0pt}
\fancyfootoffset{\marginparsep+\marginparwidth}
\newlength{\footpageshift}
\setlength{\footpageshift}
          {0.5\textwidth+0.5\marginparsep+0.5\marginparwidth-2in}
\lfoot{\hspace{\footpageshift}%
       \parbox{4in}{\, \hfill %
                    \arabic{page} of \protect\pageref*{LastPage} % +LP
%                    \arabic{page}                               % -LP
                    \hfill \,}}

% Finally, give us PDF bookmarks
\usepackage{color,hyperref}
\definecolor{darkblue}{rgb}{0.0,0.0,0.3}
\hypersetup{colorlinks,breaklinks,
            linkcolor=darkblue,urlcolor=darkblue,
            anchorcolor=darkblue,citecolor=darkblue}

%%%%%%%%%%%%%%%%%%%%%%%% End Document Setup %%%%%%%%%%%%%%%%%%%%%%%%%%%%


%%%%%%%%%%%%%%%%%%%%%%%%%%% Helper Commands %%%%%%%%%%%%%%%%%%%%%%%%%%%%

% The title (name) with a horizontal rule under it
%
% Usage: \makeheading{name}
%
% Place at top of document. It should be the first thing.
\newcommand{\makeheading}[1]%
        {\hspace*{-\marginparsep minus \marginparwidth}%
         \begin{minipage}[t]{\textwidth+\marginparwidth+\marginparsep}%
                {\large \bfseries #1}\\[-0.15\baselineskip]%
                 \rule{\columnwidth}{1pt}%
         \end{minipage}}

% The section headings
%
% Usage: \section{section name}
%
% Follow this section IMMEDIATELY with the first line of the section
% text. Do not put whitespace in between. That is, do this:
%
%       \section{My Information}
%       Here is my information.
%
% and NOT this:
%
%       \section{My Information}
%
%       Here is my information.
%
% Otherwise the top of the section header will not line up with the top
% of the section. Of course, using a single comment character (%) on
% empty lines allows for the function of the first example with the
% readability of the second example.
\renewcommand{\section}[2]%
        {\pagebreak[2]\vspace{1.3\baselineskip}%
         \phantomsection\addcontentsline{toc}{section}{#1}%
         \hspace{0in}%
         \marginpar{
         \raggedright \scshape #1}#2}

% An itemize-style list with lots of space between items
\newenvironment{outerlist}[1][\enskip\textbullet]%
        {\begin{itemize}[#1]}{\end{itemize}%
         \vspace{-.6\baselineskip}}


% An environment IDENTICAL to outerlist that has better pre-list spacing
% when used as the first thing in a \section
\newenvironment{lonelist}[1][\enskip\textbullet]%
        {\vspace{-\baselineskip}\begin{list}{#1}{%
        \setlength{\partopsep}{0pt}%
        \setlength{\topsep}{0pt}}}
        {\end{list}\vspace{-.6\baselineskip}}

% An itemize-style list with little space between items
\newenvironment{innerlist}[1][\enskip\textbullet]%
        {\begin{compactitem}[#1]}{\end{compactitem}}

% An environment IDENTICAL to innerlist that has better pre-list spacing
% when used as the first thing in a \section
\newenvironment{loneinnerlist}[1][\enskip\textbullet]%
        {\vspace{-\baselineskip}\begin{compactitem}[#1]}
        {\end{compactitem}\vspace{-.6\baselineskip}}

% To add some paragraph space between lines.
% This also tells LaTeX to preferably break a page on one of these gaps
% if there is a needed pagebreak nearby.
\newcommand{\blankline}{\quad\pagebreak[2]}

% Uses hyperref to link DOI
\newcommand\doilink[1]{\href{http://dx.doi.org/#1}{#1}}
\newcommand\doi[1]{doi:\doilink{#1}}


%%%%%%%%%%%%%%%%%%%%%%%% End Helper Commands %%%%%%%%%%%%%%%%%%%%%%%%%%%

%%%%%%%%%%%%%%%%%%%%%%%%% Begin CV Document %%%%%%%%%%%%%%%%%%%%%%%%%%%%

\begin{document}
\makeheading{Lowe Schmidt}

\section{Contact Information}
%
\newlength{\rcollength}\setlength{\rcollength}{1.85in}%
%
\begin{tabular}[t]{@{}p{\textwidth-\rcollength}p{\rcollength}}
Rubinvägen 9			& \textit{Phone:} +46 723 867 157\\
126 78 Hägersten			& \textit{E-mail:}
\href{mailto:jobs@loweschmidt.se} {jobs@loweschmidt.se}\\
\end{tabular}


\section{Experience}
%
%
\textbf{Operations Engineeer} at \href{http://www.uprise.se}{\textbf{Uprise (Electronic Arts)}}\hfill {Since July 2016}
\begin{outerlist}
\item[]

\end{outerlist}
\blankline

\textbf{Consultant} at \href{http://www.init.se}{\textbf{Init}}\hfill {November 2013 - June 2016}
\begin{outerlist}
\item[] Consultant for a small consultancy, customers in markets such as broadcasting, telecom and online gambling.

	\item Designed, automated and implemented infrastructure for Joors. Brought down the provisioning of complete environments to < 1 hour,
		using vSphere, The Foreman and Puppet.
	\item Custom Puppet module development to manage Gigaspaces, the core was also extracted (a work in progress) to an open source
		version.
	\item Developed tools to manage and fix subtitles and manifest files for the media archive at Viaplay.
	\item Automated rolling and full restarts of a Hazelcast based application using Saltstack.
\end{outerlist}
\blankline

\textbf{System Administrator} at \href{http://www.avanza.se}{\textbf{Avanza}}\hfill {April 2013 - October 2013}
\begin{outerlist}
\item[] Avanza is a bank with a focus on the nordic stock markets, the operations team is split between trading and web operations, the web
	operations team was responsible for hardware, operating systems, application servers and databases.

	\item Developed and refactored big parts of our Puppet codebase to meet language guidelines.
	\item Coached team members on idiomatic Puppet and performed code reviews.
	\item Part of the team that redesigned our datacenter, including physical and logical network planning and hardware evaluation.

\end{outerlist}
\blankline

\textbf{System Administrator} at \href{http://www.ztorm.com}{\textbf{Ztorm}}\hfill {January 2013 - March 2013}
\begin{outerlist}
\item[] System Administrator for a digital media distribution company, doing web and storage operations. 

	\item Leveraged supply\_drop to do push based automation with Puppet and Capistrano.
	\item Introduced the development team to Vagrant and automated their development environments.
\end{outerlist}
\blankline

\textbf{System Administrator} at {\textbf{The Delta Projects}}\hfill {August 2012 - December 2012}
\begin{outerlist}
	\item Part of the two man team designing the new datacenter, includin racking and stacking, physical and logical network design and
		configuration.
	\item Automated our provisioning of machines, DNS names and DHCP leases, using The Foreman. 
\end{outerlist}
\blankline

\textbf{Infrastructure Technician} at \href{http://www.unibet.com}{\textbf{Unibet}}  \hfill {March 2011 - July 2012}
\begin{outerlist}
	 \item Designed and implemented the new provisioning and management platform based on The Foreman and Puppet. Planned the release routines, versioning and wrote the majority of our first Puppet modules. Brought down our provisioning times from up to two weeks to 30 minutes.
	\item Introduced the idea of using Kanban for technical developments workflows. Appointed Kanban master.
	\item Advocated the idea of using a goalie from technical development to deflect questions regarding infrastructure, network or applications to a dedicated person on a rotating schedule.
	\item Planned and designed our new DNS infrastructure.
\end{outerlist}
\blankline

 \textbf{Systems Specialist, Consultant} at \href{http://www.24solutions.se}{\textbf{24 Solutions}}\hfill {September 2010 - March 2011}
\begin{outerlist}
	\item Introduced Kanban to the Support Services team, and taught basics of WIP limitation and lanes.
	\item Coached and taught interns the basics of Puppet as a automation platform proof of concept was being built.
\end{outerlist}
\blankline

\textbf{System Administrator} at {\textbf{Squace} \hfill {March 2010 - July 2010}
\begin{outerlist}
\item[] Squace was a mobile app for Java based feature phones

	\item Consolidated virtualisation platform from three to one.
	\item Redesigned our infrastructure in collaboration with the software architect.

\end{outerlist}
\blankline

\textbf{System Administration Consultant} at \href{http://www.r2m.se/}{\textbf{R2M}}\hfill {December 2007 - August 2009}
\begin{outerlist}	
	\item Replaced the aging Windows 2000 AD server with a new one based on Debian, OpenLDAP and Samba, included planning, procurement and
		execution.

\end{outerlist}
\blankline

\section{Education}
%
\textbf{Linux System Administration} at \href{http://www.jenseneducation.se}{\textbf{JENSEN Education}} \hfill {2005 - 2007}
\blankline

\section{Additional Engagements}
{Founder and Organizer}, \href{http://www.meetup.com/DevOps-Stockholm/}{\textbf{DevOps Stockholm}} \hfill{Since March 2011}
\begin{outerlist} 
	\item[] DevOps Stockholm is a user group for people interested in things like; infrastructure as code, cross functional teams,
		continous delivery and optimizing for happiness in the organization. There are currently over 1000 members.
\end{outerlist}
\blankline

{Co-Founder and Organizer}, \href{http://bodyfest.se} {\textbf{Bodyfest}} \hfill{January 2010 - December 2013}
\begin{outerlist}
	\item[] Bodyfest is an annual music event with a focus on electronic music from or inspired by the 80's. Bands such as Front
		242, Pouppée Fabrikk, The Klinik and The Invincible Spirit have performed, up to 800 people attend.
\end{outerlist}
\blankline

{Organizer and DJ}, {\textbf{Club Bodytåget}}, \hfill{August 2008 - December 2013}
\begin{outerlist}
\item[] Club Bodytåget is a regular live music club hosted at Nalen in Stockholm. Tasks included budgeting, band contact and DJ:ing.
\end{outerlist}

\section{References}
Available upon request.

\section{Languages}
Fluent in Swedish, English and German.
\end{document}

%%%%%%%%%%%%%%%%%%%%%%%%%% End CV Document %%%%%%%%%%%%%%%%%%%%%%%%%%%%%
