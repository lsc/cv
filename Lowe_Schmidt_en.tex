\documentclass[8pt]{article}

% This is a helpful package that puts math inside length specifications
\usepackage{calc}
\usepackage[latin1]{inputenc}
\usepackage[swedish]{babel}
\usepackage[T1]{fontenc}
\usepackage{tgpagella}
\usepackage{eurosym}


% Simpler bibsection for CV sections
% (thanks to natbib for inspiration)
\makeatletter
\newlength{\bibhang}
\setlength{\bibhang}{1em}
\newlength{\bibsep}
 {\@listi \global\bibsep\itemsep \global\advance\bibsep by\parsep}
\newenvironment{bibsection}
    {\minipage[t]{\linewidth}\list{}{%
        \setlength{\leftmargin}{\bibhang}%
        \setlength{\itemindent}{-\leftmargin}%
        \setlength{\itemsep}{\bibsep}%
        \setlength{\parsep}{\z@}%
        }}
    {\endlist\endminipage}
\makeatother

% Layout: Puts the section titles on left side of page
\reversemarginpar

%
%         PAPER SIZE, PAGE NUMBER, AND DOCUMENT LAYOUT NOTES:
%
% The next \usepackage line changes the layout for CV style section
% headings as marginal notes. It also sets up the paper size as either
% letter or A4. By default, letter was used. If A4 paper is desired,
% comment out the letterpaper lines and uncomment the a4paper lines.
%
% As you can see, the margin widths and section title widths can be
% easily adjusted.
%
% ALSO: Notice that the includefoot option can be commented OUT in order
% to put the PAGE NUMBER *IN* the bottom margin. This will make the
% effective text area larger.
%
% IF YOU WISH TO REMOVE THE ``of LASTPAGE'' next to each page number,
% see the note about the +LP and -LP lines below. Comment out the +LP
% and uncomment the -LP.
%
% IF YOU WISH TO REMOVE PAGE NUMBERS, be sure that the includefoot line
% is uncommented and ALSO uncomment the \pagestyle{empty} a few lines
% below.
%

%% Use these lines for letter-sized paper
%\usepackage[paper=letterpaper,
%            %includefoot, % Uncomment to put page number above margin
%            marginparwidth=1.2in,     % Length of section titles
%            marginparsep=.05in,       % Space between titles and text
%            margin=1in,               % 1 inch margins
%            includemp]{geometry}

% Use these lines for A4-sized paper
\usepackage[paper=a4paper,
            %includefoot, % Uncomment to put page number above margin
            marginparwidth=30.5mm,    % Length of section titles
            marginparsep=1.5mm,       % Space between titles and text
            margin=25mm,              % 25mm margins
            includemp]{geometry}

%% More layout: Get rid of indenting throughout entire document
\setlength{\parindent}{0in}

%% This gives us fun enumeration environments. compactitem will be nice.
\usepackage{paralist}

%% Reference the last page in the page number
%
% NOTE: comment the +LP line and uncomment the -LP line to have page
%       numbers without the ``of ##'' last page reference)
%
% NOTE: uncomment the \pagestyle{empty} line to get rid of all page
%       numbers (make sure includefoot is commented out above)
%
\usepackage{fancyhdr,lastpage}
\pagestyle{fancy}
%\pagestyle{empty}      % Uncomment this to get rid of page numbers
\fancyhf{}\renewcommand{\headrulewidth}{0pt}
\fancyfootoffset{\marginparsep+\marginparwidth}
\newlength{\footpageshift}
\setlength{\footpageshift}
          {0.5\textwidth+0.5\marginparsep+0.5\marginparwidth-2in}
\lfoot{\hspace{\footpageshift}%
       \parbox{4in}{\, \hfill %
                    \arabic{page} of \protect\pageref*{LastPage} % +LP
%                    \arabic{page}                               % -LP
                    \hfill \,}}

% Finally, give us PDF bookmarks
\usepackage{color,hyperref}
\definecolor{darkblue}{rgb}{0.0,0.0,0.3}
\hypersetup{colorlinks,breaklinks,
            linkcolor=darkblue,urlcolor=darkblue,
            anchorcolor=darkblue,citecolor=darkblue}

%%%%%%%%%%%%%%%%%%%%%%%% End Document Setup %%%%%%%%%%%%%%%%%%%%%%%%%%%%


%%%%%%%%%%%%%%%%%%%%%%%%%%% Helper Commands %%%%%%%%%%%%%%%%%%%%%%%%%%%%

% The title (name) with a horizontal rule under it
%
% Usage: \makeheading{name}
%
% Place at top of document. It should be the first thing.
\newcommand{\makeheading}[1]%
        {\hspace*{-\marginparsep minus \marginparwidth}%
         \begin{minipage}[t]{\textwidth+\marginparwidth+\marginparsep}%
                {\large \bfseries #1}\\[-0.15\baselineskip]%
                 \rule{\columnwidth}{1pt}%
         \end{minipage}}

% The section headings
%
% Usage: \section{section name}
%
% Follow this section IMMEDIATELY with the first line of the section
% text. Do not put whitespace in between. That is, do this:
%
%       \section{My Information}
%       Here is my information.
%
% and NOT this:
%
%       \section{My Information}
%
%       Here is my information.
%
% Otherwise the top of the section header will not line up with the top
% of the section. Of course, using a single comment character (%) on
% empty lines allows for the function of the first example with the
% readability of the second example.
\renewcommand{\section}[2]%
        {\pagebreak[2]\vspace{1.3\baselineskip}%
         \phantomsection\addcontentsline{toc}{section}{#1}%
         \hspace{0in}%
         \marginpar{
         \raggedright \scshape #1}#2}

% An itemize-style list with lots of space between items
\newenvironment{outerlist}[1][\enskip\textbullet]%
        {\begin{itemize}[#1]}{\end{itemize}%
         \vspace{-.6\baselineskip}}


% An environment IDENTICAL to outerlist that has better pre-list spacing
% when used as the first thing in a \section
\newenvironment{lonelist}[1][\enskip\textbullet]%
        {\vspace{-\baselineskip}\begin{list}{#1}{%
        \setlength{\partopsep}{0pt}%
        \setlength{\topsep}{0pt}}}
        {\end{list}\vspace{-.6\baselineskip}}

% An itemize-style list with little space between items
\newenvironment{innerlist}[1][\enskip\textbullet]%
        {\begin{compactitem}[#1]}{\end{compactitem}}

% An environment IDENTICAL to innerlist that has better pre-list spacing
% when used as the first thing in a \section
\newenvironment{loneinnerlist}[1][\enskip\textbullet]%
        {\vspace{-\baselineskip}\begin{compactitem}[#1]}
        {\end{compactitem}\vspace{-.6\baselineskip}}

% To add some paragraph space between lines.
% This also tells LaTeX to preferably break a page on one of these gaps
% if there is a needed pagebreak nearby.
\newcommand{\blankline}{\quad\pagebreak[2]}

% Uses hyperref to link DOI
\newcommand\doilink[1]{\href{http://dx.doi.org/#1}{#1}}
\newcommand\doi[1]{doi:\doilink{#1}}


%%%%%%%%%%%%%%%%%%%%%%%% End Helper Commands %%%%%%%%%%%%%%%%%%%%%%%%%%%

%%%%%%%%%%%%%%%%%%%%%%%%% Begin CV Document %%%%%%%%%%%%%%%%%%%%%%%%%%%%

\begin{document}
\makeheading{Lowe Schmidt}

\section{Contact Information}
%
\newlength{\rcollength}\setlength{\rcollength}{1.85in}%
%
\begin{tabular}[t]{@{}p{\textwidth-\rcollength}p{\rcollength}}
Rubinvägen 9			& \textit{Phone:} +46 723 867 157\\
126 78 Hägersten			& \textit{E-mail:}
\href{mailto:jobs@loweschmidt.se} {jobs@loweschmidt.se}\\
\end{tabular}


\section{Experience}
%
%
\textbf{Technical Consultant} at \href{http://www.init.se}{\textbf{Init}}\hfill {Since November 2013}
\begin{outerlist}
\item[]  As a Consultant for Init I have worked with customers in the media and broadcasting business, startups in the telecom sphere and online gambling sites.

	\item Designed and implemented a complete infrastructure including provisioning, configuration management, ad-hoc command execution, monitoring, metrics collection and log aggregation. Custom Puppet module development for automating application server installs. Enabled us to spin up new environments on the vSphere based cluster in less than one hour.




 \textit{Technology keywords:} CentOS, Bash, GFS, VMWare, Puppet, The Foreman, Saltstack, Graphite, Grafana, ELK Stack, Gigaspaces XAP 
\end{outerlist}
\blankline

\textbf{System Administrator} at \href{http://www.avanza.se}{\textbf{Avanza}}\hfill {April 2013 - October 2013}
\begin{outerlist}
\item[] At Avanza I worked with 



\textit{Technology keywords:} RHEL, Splunk, Puppet, Oracle, MySQL, Tomcat, Gigaspaces XAP,
\end{outerlist}
\blankline

\textbf{System Administrator} at \href{http://www.ztorm.com}{\textbf{Ztorm}}\hfill {January 2013 - March 2013}
\begin{outerlist}
\item[] System Administrator for a digital media distribution company, doing web and storage operations. 


\textit{Technology keywords:} Gentoo, Apache, PHP, OpenSolaris, FreeBSD, OpenBSD, ZFS.
\end{outerlist}
\blankline

\textbf{System Administrator} at {\textbf{The Delta Projects}}\hfill {August 2012 - December 2012}
\begin{outerlist}
\item[] Delta Projects provides a ad serving, real time bidding and analytics platform.

\item[]I was part of the design and implementation of the new datacenter to expand into the Netherlands and Germany. Automated our provisioning of machines DNS names and IP adresses, racked and stacked machines, designed networks, configured switches and wrote chef cookbooks.

\textit{Technology keywords:} Ubuntu, Jetty, MySQL, Chef, The Foreman, BIND, XenServer, Force10 switches. 
\end{outerlist}
\blankline

\textbf{Infrastructure Technician} at \href{http://www.unibet.com}{\textbf{Unibet}}  \hfill {March 2011 - July 2012}
\begin{outerlist}
	\item[]I was part of the infrastructure team within the technical development division at Unibet, 
	 \item Designed and implemented the new provisioning and management platform based on The Foreman and Puppet. I planned our release routines, versioning and wrote the majority of our first Puppet modules.
	\item Introduced the idea of using Kanban for technical developments workflows, I was also appointed Kanban master.
	\item Advocated the idea of using a goalie from technical development to deflect questions regarding infrastructure, network or applications to a dedicated person on a rotating schedule.

\textit{Technology keywords:} RHEL, Ubuntu, F5 BigIP, HDS SAN, Puppet, The Foreman, MCollective, BIND, Glassfish, MySQL, Oracle RAC.
	
\end{outerlist}
\blankline

 \textbf{Systems Specialist, Consultant} at \href{http://www.24solutions.se}{\textbf{24 Solutions}}\hfill {September 2010 - March 2011}
\begin{outerlist}
	\item[] Consulted at Bwin Games AB in the Support Services team within Application Management, where I worked with the monitoring and metrics gathering platform as well as log aggregation. I also took part in the Puppet pilot project and introduced Kanban as a concept and work process for the Support Services team.

\textit{Technology Keywords:} RHEL, Nagios, Cacti, rsyslog, Puppet.
\end{outerlist}
\blankline

\textbf{System Administrator} at {\textbf{Squace} \hfill {March 2010 - July 2010}
\begin{outerlist}
	\item Squace was a mobile app for Java based "dumb" phones for accessing the web and with an app concept. 

	\item[] As the sole systems administrator for a small startup I did most of everything, procurement, infrastructure design, virtualization consolidation, minor SQL development, evaluating databases and fixing paper jams. Squace failed to secure funding and ceased operations in Sweden in August 2010.
 
	\textit{Technology keywords:} CentOS, Xen, Postgres, JVM, Tomcat, MongoDB.
\end{outerlist}
\blankline

\textbf{System Administration Consultant} at \href{http://www.r2m.se/}{\textbf{R2M}}\hfill {December 2007 - August 2009}
\begin{outerlist}	
	\item[] I planned and executed the migration work from Microsoft Windows AD to a FLOSS replacement based on Debian, OpenLDAP and Samba. Client work included National Swedish Televison (SVT) where I worked on their java based integration platform and the union Ledarna where I did operations for their Websphere based web platform.

\textit{Technology Keywords:} WebLogic Integration, Debian, Samba, OpenLdap, WebSphere Express, Windows 2003 Server.
\end{outerlist}
\blankline

\section{Education}
%
\textbf{Linux System Administration} at \href{http://www.jenseneducation.se}{\textbf{JENSEN Education}} \hfill {2005 - 2007}
\begin{outerlist}
	\item[] Vocational degree in Linux system administration. I spent a third of the program as an intern at two different companies, Avalanche Studios and AlaTEST.
\end{outerlist}
\blankline

\section{Technology}
Excellent knowledge of UNIX like operating systems, good networking skills; including load balancers and switches, extensive experience with configuration management tools especially Puppet but also Chef and Saltstack, decent scripting skills with a favour for Perl and Ruby, exposed to both RDBMS and NoSQL datastores including MySQL, PostgreSQL, MongoDB, Redis, ElasticSearch.
\section{Additional Engagements}
{Founder and Organizer}, \href{http://www.meetup.com/DevOps-Stockholm/}{\textbf{DevOps Stockholm}} \hfill{Since March 2011}
\begin{outerlist} 
	\item[] DevOps Stockholm is a user group for people interested in things like; infrastructure as code, cross functional teams , continous delivery and optimizing for happiness in the organization. I have presented on topics such as Puppet and Mcollective. We have over 1000 members in the group. 
\end{outerlist}
\blankline

{Co-Founder and Organizer}, \href{http://bodyfest.se} {\textbf{Bodyfest}} \hfill{January 2010 - December 2013}
\begin{outerlist}
	\item[] Bodyfest is an annual music event with a focus on electronic music from or inspired by the 80's. We have had band such as Front 242, Pouppée Fabrikk, The Klinik and The Invincible Spirit perform and have an average of 500 visitors each year. 
\end{outerlist}
\blankline

{Organizer and DJ}, {\textbf{Club Bodytåget}}, \hfill{August 2008 - December 2013}
\begin{outerlist}
\item[] Club Bodytåget (lit. The body train) is a regular live music club hosted at Nalen in Stockholm. I help organize the events including band contact, budgeting and from time to time I play some records.
\end{outerlist}

\section{References}
Available upon request.

\section{Languages}
Fluent in Swedish, English and German.
\end{document}

%%%%%%%%%%%%%%%%%%%%%%%%%% End CV Document %%%%%%%%%%%%%%%%%%%%%%%%%%%%%
