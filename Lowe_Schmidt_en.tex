\documentclass[8pt]{article}

% This is a helpful package that puts math inside length specifications
\usepackage{calc}
\usepackage[latin1]{inputenc}
\usepackage[swedish]{babel}
%%\usepackage[T1]{fontenc}
\usepackage{tgpagella}
\usepackage{eurosym}


% Simpler bibsection for CV sections
% (thanks to natbib for inspiration)
\makeatletter
\newlength{\bibhang}
\setlength{\bibhang}{1em}
\newlength{\bibsep}
 {\@listi \global\bibsep\itemsep \global\advance\bibsep by\parsep}
\newenvironment{bibsection}
    {\minipage[t]{\linewidth}\list{}{%
        \setlength{\leftmargin}{\bibhang}%
        \setlength{\itemindent}{-\leftmargin}%
        \setlength{\itemsep}{\bibsep}%
        \setlength{\parsep}{\z@}%
        }}
    {\endlist\endminipage}
\makeatother

% Layout: Puts the section titles on left side of page
\reversemarginpar

%
%         PAPER SIZE, PAGE NUMBER, AND DOCUMENT LAYOUT NOTES:
%
% The next \usepackage line changes the layout for CV style section
% headings as marginal notes. It also sets up the paper size as either
% letter or A4. By default, letter was used. If A4 paper is desired,
% comment out the letterpaper lines and uncomment the a4paper lines.
%
% As you can see, the margin widths and section title widths can be
% easily adjusted.
%
% ALSO: Notice that the includefoot option can be commented OUT in order
% to put the PAGE NUMBER *IN* the bottom margin. This will make the
% effective text area larger.
%
% IF YOU WISH TO REMOVE THE ``of LASTPAGE'' next to each page number,
% see the note about the +LP and -LP lines below. Comment out the +LP
% and uncomment the -LP.
%
% IF YOU WISH TO REMOVE PAGE NUMBERS, be sure that the includefoot line
% is uncommented and ALSO uncomment the \pagestyle{empty} a few lines
% below.
%

%% Use these lines for letter-sized paper
%\usepackage[paper=letterpaper,
%            %includefoot, % Uncomment to put page number above margin
%            marginparwidth=1.2in,     % Length of section titles
%            marginparsep=.05in,       % Space between titles and text
%            margin=1in,               % 1 inch margins
%            includemp]{geometry}

% Use these lines for A4-sized paper
\usepackage[paper=a4paper,
            %includefoot, % Uncomment to put page number above margin
            marginparwidth=30.5mm,    % Length of section titles
            marginparsep=1.5mm,       % Space between titles and text
            margin=25mm,              % 25mm margins
            includemp]{geometry}

%% More layout: Get rid of indenting throughout entire document
\setlength{\parindent}{0in}

%% This gives us fun enumeration environments. compactitem will be nice.
\usepackage{paralist}

%% Reference the last page in the page number
%
% NOTE: comment the +LP line and uncomment the -LP line to have page
%       numbers without the ``of ##'' last page reference)
%
% NOTE: uncomment the \pagestyle{empty} line to get rid of all page
%       numbers (make sure includefoot is commented out above)
%
\usepackage{fancyhdr,lastpage}
\pagestyle{fancy}
%\pagestyle{empty}      % Uncomment this to get rid of page numbers
\fancyhf{}\renewcommand{\headrulewidth}{0pt}
\fancyfootoffset{\marginparsep+\marginparwidth}
\newlength{\footpageshift}
\setlength{\footpageshift}
          {0.5\textwidth+0.5\marginparsep+0.5\marginparwidth-2in}
\lfoot{\hspace{\footpageshift}%
       \parbox{4in}{\, \hfill %
                    \arabic{page} of \protect\pageref*{LastPage} % +LP
%                    \arabic{page}                               % -LP
                    \hfill \,}}

% Finally, give us PDF bookmarks
\usepackage{color,hyperref}
\definecolor{darkblue}{rgb}{0.0,0.0,0.3}
\hypersetup{colorlinks,breaklinks,
            linkcolor=darkblue,urlcolor=darkblue,
            anchorcolor=darkblue,citecolor=darkblue}

%%%%%%%%%%%%%%%%%%%%%%%% End Document Setup %%%%%%%%%%%%%%%%%%%%%%%%%%%%


%%%%%%%%%%%%%%%%%%%%%%%%%%% Helper Commands %%%%%%%%%%%%%%%%%%%%%%%%%%%%

% The title (name) with a horizontal rule under it
%
% Usage: \makeheading{name}
%
% Place at top of document. It should be the first thing.
\newcommand{\makeheading}[1]%
        {\hspace*{-\marginparsep minus \marginparwidth}%
         \begin{minipage}[t]{\textwidth+\marginparwidth+\marginparsep}%
                {\large \bfseries #1}\\[-0.15\baselineskip]%
                 \rule{\columnwidth}{1pt}%
         \end{minipage}}

% The section headings
%
% Usage: \section{section name}
%
% Follow this section IMMEDIATELY with the first line of the section
% text. Do not put whitespace in between. That is, do this:
%
%       \section{My Information}
%       Here is my information.
%
% and NOT this:
%
%       \section{My Information}
%
%       Here is my information.
%
% Otherwise the top of the section header will not line up with the top
% of the section. Of course, using a single comment character (%) on
% empty lines allows for the function of the first example with the
% readability of the second example.
\renewcommand{\section}[2]%
        {\pagebreak[2]\vspace{1.3\baselineskip}%
         \phantomsection\addcontentsline{toc}{section}{#1}%
         \hspace{0in}%
         \marginpar{
         \raggedright \scshape #1}#2}

% An itemize-style list with lots of space between items
\newenvironment{outerlist}[1][\enskip\textbullet]%
        {\begin{itemize}[#1]}{\end{itemize}%
         \vspace{-.6\baselineskip}}


% An environment IDENTICAL to outerlist that has better pre-list spacing
% when used as the first thing in a \section
\newenvironment{lonelist}[1][\enskip\textbullet]%
        {\vspace{-\baselineskip}\begin{list}{#1}{%
        \setlength{\partopsep}{0pt}%
        \setlength{\topsep}{0pt}}}
        {\end{list}\vspace{-.6\baselineskip}}

% An itemize-style list with little space between items
\newenvironment{innerlist}[1][\enskip\textbullet]%
        {\begin{compactitem}[#1]}{\end{compactitem}}

% An environment IDENTICAL to innerlist that has better pre-list spacing
% when used as the first thing in a \section
\newenvironment{loneinnerlist}[1][\enskip\textbullet]%
        {\vspace{-\baselineskip}\begin{compactitem}[#1]}
        {\end{compactitem}\vspace{-.6\baselineskip}}

% To add some paragraph space between lines.
% This also tells LaTeX to preferably break a page on one of these gaps
% if there is a needed pagebreak nearby.
\newcommand{\blankline}{\quad\pagebreak[2]}

% Uses hyperref to link DOI
\newcommand\doilink[1]{\href{http://dx.doi.org/#1}{#1}}
\newcommand\doi[1]{doi:\doilink{#1}}


%%%%%%%%%%%%%%%%%%%%%%%% End Helper Commands %%%%%%%%%%%%%%%%%%%%%%%%%%%

%%%%%%%%%%%%%%%%%%%%%%%%% Begin CV Document %%%%%%%%%%%%%%%%%%%%%%%%%%%%

\begin{document}
\makeheading{Lowe Schmidt}

\section{Contact Information}
%
\newlength{\rcollength}\setlength{\rcollength}{1.85in}%
%
\begin{tabular}[t]{@{}p{\textwidth-\rcollength}p{\rcollength}}
Jarl Kulles Gata 6			& \textit{Phone:} +46 768 875 723\\
126 54 Hägersten			& \textit{E-mail:}
\href{mailto:jobs@loweschmidt.se} {jobs@loweschmidt.se}\\
\end{tabular}


\section{Experience}
%
%
\textbf{System Administrator} at \href{http://www.avanza.se}{\textbf{Avanza}}\hfill {Since April 2013}
\begin{outerlist}
\item[] The platform group within Avanza is responsible for web and trade operations at Avanza. My focus lies mainly with the day to day and long term administration of the web platform running on Red Hat Linux servers with Gigaspaces technology at the application layer. As my interest in automation and orchestration are strong I also try to take extra care of our Puppet and Mcollective instances, release strategies for Puppet code, upgrade paths and providing feedback on manifest development to co-workers.
\end{outerlist}
\blankline

\textbf{System Administrator} at \href{http://www.ztorm.com}{\textbf{Ztorm}}\hfill {January 2013 - March 2013}
\begin{outerlist}
\item[] System administration of our LAMP stack based on Gentoo Linux and support systems based on OpenSolaris, FreeBSD and OpenBSD. I also performed some benchmark and performance testing of hardware.
\end{outerlist}
\blankline

\textbf{System Administrator} at {\textbf{The Delta Projects}}\hfill {August 2012 - December 2012}
\begin{outerlist}
\item[] Scaling out our real time bidding and ad-serving platform to handle customers and impressions from Germany and the Netherlands. I handled most aspects of our IT operations from racking and stacking machines, refine network architecture and write chef cookbooks to automate our infrastructure. I also introduced The Foreman for provisioning, DNS records and DCHP leases automation. 
\end{outerlist}
\blankline

\textbf{Infrastructure Technician} at \href{http://www.unibet.com}{\textbf{Unibet}}  \hfill {March 2011 - July 2012}
\begin{outerlist}
	\item[] As part of the Technical Development Infastructure Team my focus was on operating systems, load balancing, storage and maintenance of a 1000 machines strong server park. I also took an active part in routine and process development. 
	 \item Designed and implemented the new provisioning and management platform based on The Foreman and Puppet. I planned our release routines, versioning and wrote the bulk of our first Puppet modules.
	\item Introduced the idea of using Kanban for TD workflows, I was also appointed Kanban master.
	\item Advocated the idea of using a goalie from TD to deflect questions regarding infrastructure, network or applications to a dedicated person on a rotating schedule.
	
\end{outerlist}
\blankline

 \textbf{Systems Specialist} at \href{http://www.24solutions.se}{\textbf{24 Solutions}}\hfill {September 2010 - March 2011}
\begin{outerlist}
	\item[] Consulting for a big online gambling company with focus on monitoring, metrics and centralized logging. Introduced kanban as a proof of concept for the team.
\end{outerlist}
\blankline

\textbf{System Administrator} at {\textbf{Squace} \hfill {March 2010 - July 2010}
\begin{outerlist}
	\item[] Sole system administrator for a startup, did everything from infrastructure design, to fix the printer, write simple SQL and procure new workstations for the development and qa teams. At Squace I discovered Puppet and my interest in datacenter automation and orchestration. Squace ran out of money and ceased operations in August 2010.
\end{outerlist}
\blankline

\textbf{System Administrator} at \href{http://www.r2m.se/}{\textbf{R2M}}\hfill {December 2007 - August 2009}
\begin{outerlist}	
	\item[] Inhouse system administration and consulting on the Java platform, the first project was to replace our Windows 2000 Active Directory server with one based on Debian, OpenLDAP and Samba. WebLogic Integration consulting at SVT and WebSphere consulting at Ledarna.
\end{outerlist}
\blankline

\section{Education}
%
\textbf{Linux System Administration} at \href{http://www.jenseneducation.se}{\textbf{JENSEN Education}} \hfill {2005 - 2007}
\begin{outerlist}
	\item[] Vocational degree in Linux system administration. I spent a third of the program as an intern at two different companies, Avalanche Studios and AlaTEST.
\end{outerlist}
\blankline

\section{Technology}
GNU/Linux, TCP/IP, DNS, DHCP, PXE, Keepalived, Puppet, The Foreman, F5 BigIP, nginx, Apache, Ruby, Perl, Bash, MySQL, PostgreSQL, MongoDB, Nagios, Git, Jira, Confluence.
 
\section{Additional Engagements}
{Founder and Organizer}, \href{http://www.meetup.com/DevOps-Stockholm/}{\textbf{Stockholm DevOps}} \hfill{Since March 2011}
\begin{outerlist} 
	\item[] Stockholm DevOps is local user group for people interested in areas such as; infrastructure as code, cross functional teams between developers and operations, continous delivery and optimizing for happiness in the organization. I have presented on topics such as Puppet and Mcollective. Our meetup group has over 400 members to date. 
\end{outerlist}
\blankline

{Co-Founder and Organizer}, \href{http://bodyfest.se} {\textbf{Bodyfest}} \hfill{Since January 2010}
\begin{outerlist}
	\item[] Bodyfest is an annual music event with a focus on electronic music from or inspired by the 80's. We have had band such as Front 242, Pouppée Fabrikk, The Klinik and The Invincible Spirit perform and have an average of 500 visitors each year. 
\end{outerlist}
\blankline

{Organizer and DJ}, {\textbf{Club Bodytåget}}, \hfill{Since August 2008}
\begin{outerlist}
\item[] Club Bodytåget (lit. The body train) is a regular live music club hosted at Nalen in Stockholm. I help organize the events including band contact, budgeting and from time to time I play some records.
\end{outerlist}

\section{References}
Available upon request.

\section{Languages}
Fluent in Swedish, English and German.
\end{document}

%%%%%%%%%%%%%%%%%%%%%%%%%% End CV Document %%%%%%%%%%%%%%%%%%%%%%%%%%%%%
