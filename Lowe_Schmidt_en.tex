\documentclass[10pt,sans]{moderncv}

% This is a helpful package that puts math inside length specifications
\usepackage{calc}
%\usepackage[latin1]{inputenc}
\usepackage[swedish]{babel}
\usepackage[T1]{fontenc}
\usepackage{tgpagella}
\usepackage{eurosym}
%\usepackage{fontspec}
	%\setmainfont{Candara}

% Simpler bibsection for CV sections
% (thanks to natbib for inspiration)
\makeatletter
\newlength{\bibhang}
\setlength{\bibhang}{1em}
\newlength{\bibsep}
 {\@listi \global\bibsep\itemsep \global\advance\bibsep by\parsep}
\newenvironment{bibsection}
    {\minipage[t]{\linewidth}\list{}{%
        \setlength{\leftmargin}{\bibhang}%
        \setlength{\itemindent}{-\leftmargin}%
        \setlength{\itemsep}{\bibsep}%
        \setlength{\parsep}{\z@}%
        }}
    {\endlist\endminipage}
\makeatother

% Layout: Puts the section titles on left side of page
\reversemarginpar

%
%         PAPER SIZE, PAGE NUMBER, AND DOCUMENT LAYOUT NOTES:
%
% The next \usepackage line changes the layout for CV style section
% headings as marginal notes. It also sets up the paper size as either
% letter or A4. By default, letter was used. If A4 paper is desired,
% comment out the letterpaper lines and uncomment the a4paper lines.
%
% As you can see, the margin widths and section title widths can be
% easily adjusted.
%
% ALSO: Notice that the includefoot option can be commented OUT in order
% to put the PAGE NUMBER *IN* the bottom margin. This will make the
% effective text area larger.
%
% IF YOU WISH TO REMOVE THE ``of LASTPAGE'' next to each page number,
% see the note about the +LP and -LP lines below. Comment out the +LP
% and uncomment the -LP.
%
% IF YOU WISH TO REMOVE PAGE NUMBERS, be sure that the includefoot line
% is uncommented and ALSO uncomment the \pagestyle{empty} a few lines
% below.
%

%% Use these lines for letter-sized paper
%\usepackage[paper=letterpaper,
%            %includefoot, % Uncomment to put page number above margin
%            marginparwidth=1.2in,     % Length of section titles
%            marginparsep=.05in,       % Space between titles and text
%            margin=1in,               % 1 inch margins
%            includemp]{geometry}

% Use these lines for A4-sized paper
\usepackage[paper=a4paper,
            includefoot, % Uncomment to put page number above margin
            marginparwidth=30.5mm,    % Length of section titles
            marginparsep=1.5mm,       % Space between titles and text
            margin=25mm,              % 25mm margins
            includemp]{geometry}

%% More layout: Get rid of indenting throughout entire document
\setlength{\parindent}{0in}

%% This gives us fun enumeration environments. compactitem will be nice.
\usepackage{paralist}

%% Reference the last page in the page number
%
% NOTE: comment the +LP line and uncomment the -LP line to have page
%       numbers without the ``of ##'' last page reference)
%
% NOTE: uncomment the \pagestyle{empty} line to get rid of all page
%       numbers (make sure includefoot is commented out above)
%
\usepackage{fancyhdr,lastpage}
\pagestyle{fancy}
%\pagestyle{empty}      % Uncomment this to get rid of page numbers
\fancyhf{}\renewcommand{\headrulewidth}{0pt}
\fancyfootoffset{\marginparsep+\marginparwidth}
\newlength{\footpageshift}
\setlength{\footpageshift}
          {0.5\textwidth+0.5\marginparsep+0.5\marginparwidth-2in}
\lfoot{\hspace{\footpageshift}%
       \parbox{4in}{\, \hfill %
                    \arabic{page} of \protect\pageref*{LastPage} % +LP
%                    \arabic{page}                               % -LP
                    \hfill \,}}

% Finally, give us PDF bookmarks
\usepackage{color,hyperref}
\definecolor{darkblue}{rgb}{0.0,0.0,0.3}
\hypersetup{colorlinks,breaklinks,
            linkcolor=darkblue,urlcolor=darkblue,
            anchorcolor=darkblue,citecolor=darkblue}

%%%%%%%%%%%%%%%%%%%%%%%% End Document Setup %%%%%%%%%%%%%%%%%%%%%%%%%%%%


%%%%%%%%%%%%%%%%%%%%%%%%%%% Helper Commands %%%%%%%%%%%%%%%%%%%%%%%%%%%%

% The title (name) with a horizontal rule under it
%
% Usage: \makeheading{name}
%
% Place at top of document. It should be the first thing.
\newcommand{\makeheading}[1]%
        {\hspace*{-\marginparsep minus \marginparwidth}%
         \begin{minipage}[t]{\textwidth+\marginparwidth+\marginparsep}%
                {\large \bfseries #1}\\[-0.15\baselineskip]%
                 \rule{\columnwidth}{1pt}%
         \end{minipage}}

% The section headings
%
% Usage: \section{section name}
%
% Follow this section IMMEDIATELY with the first line of the section
% text. Do not put whitespace in between. That is, do this:
%
%       \section{My Information}
%       Here is my information.
%
% and NOT this:
%
%       \section{My Information}
%
%       Here is my information.
%
% Otherwise the top of the section header will not line up with the top
% of the section. Of course, using a single comment character (%) on
% empty lines allows for the function of the first example with the
% readability of the second example.
\renewcommand{\section}[2]%
        {\pagebreak[2]\vspace{1.3\baselineskip}%
         \phantomsection\addcontentsline{toc}{section}{#1}%
         \hspace{0in}%
         \marginpar{
         \raggedright \scshape #1}#2}

% An itemize-style list with lots of space between items
\newenvironment{outerlist}[1][\enskip\textbullet]%
        {\begin{itemize}[#1]}{\end{itemize}%
         \vspace{-.6\baselineskip}}


% An environment IDENTICAL to outerlist that has better pre-list spacing
% when used as the first thing in a \section
\newenvironment{lonelist}[1][\enskip\textbullet]%
        {\vspace{-\baselineskip}\begin{list}{#1}{%
        \setlength{\partopsep}{0pt}%
        \setlength{\topsep}{0pt}}}
        {\end{list}\vspace{-.6\baselineskip}}

% An itemize-style list with little space between items
\newenvironment{innerlist}[1][\enskip\textbullet]%
        {\begin{compactitem}[#1]}{\end{compactitem}}

% An environment IDENTICAL to innerlist that has better pre-list spacing
% when used as the first thing in a \section
\newenvironment{loneinnerlist}[1][\enskip\textbullet]%
        {\vspace{-\baselineskip}\begin{compactitem}[#1]}
        {\end{compactitem}\vspace{-.6\baselineskip}}

% To add some paragraph space between lines.
% This also tells LaTeX to preferably break a page on one of these gaps
% if there is a needed pagebreak nearby.
\newcommand{\blankline}{\quad\pagebreak[2]}

% Uses hyperref to link DOI
\newcommand\doilink[1]{\href{http://dx.doi.org/#1}{#1}}
\newcommand\doi[1]{doi:\doilink{#1}}


%%%%%%%%%%%%%%%%%%%%%%%% End Helper Commands %%%%%%%%%%%%%%%%%%%%%%%%%%%

%%%%%%%%%%%%%%%%%%%%%%%%% Begin CV Document %%%%%%%%%%%%%%%%%%%%%%%%%%%%

\begin{document}
\makeheading{Lowe Schmidt}

\section{Contact Information}
%
\newlength{\rcollength}\setlength{\rcollength}{1.85in}%
%
\begin{tabular}[t]{@{}p{\textwidth-\rcollength}p{\rcollength}}
Rubinvägen 9			& \textit{Phone:} +46 723 867 157\\
126 78 Hägersten		& \textit{E-mail:} \href{mailto:jobs@loweschmidt.se} {jobs@loweschmidt.se}\\

\end{tabular}

\section{Experience}
%
%
\textbf{Senior Operations Engineer} at \href{https://www.qapital.com}{\textbf{Qapital}}\hfill {Since September 2017}
\begin{outerlist}
\item[] Qapital is a neobank offering saving, spending and investment products together with a unique rules and goals approach to finances. 
\item[] In the infrastructure team am I responsible for infrastructure architecture, design and development, fully hosted in Amazon Web Services.

\item Transformed a manually provisioned and static compute platform with lots of single point of failures to a automated, distributed platform based around the HashiStack (Consul, Nomad, Vault) and provisioned with Terraform and Ansible. The platform went live in production November 2018 with only minor service disruptions.
\item Migrated manually managed EC2 instances running Percona Galera cluster to AWS Aurora, cutting down maintenance work from 10 of hours a week to minutes a month. Moved TB's of data with no noticeable user impact.
\item Automating and codifying repetitive work like certificate issuing, platform creation and destruction, AMI building and deployment of services.
 
\end{outerlist}
\blankline

\textbf{Operations Engineeer} at \href{http://www.dice.se}{\textbf{DICE}}\hfill {July 2016 - September 2017}
\begin{outerlist}
\item[]I am part of the Infrastructure and Developer Experience team where we simplify and enable developers to be more productive, with more reliability and with less overhead. We work with cloud providers, container orchestration, configuration management and custom tools development.

\item Operated and developed the infrastructure that ran all financial transactions for Battlefield and Battlefront titles, lots of ASG in AWS, a bunch of AWS Aurora clusters, lots and lots of SNS and SQS. All running on Apache Mesos, Apache Aurora, built on Scala, Finagle and Thrift.
\item Collaborated with EA in the design and implementation of a new reference platform built with Terraform running Kubernetes. 
\item Automated the creation of Team City build agents running on Kubernetes, implemented it with Helm.
\end{outerlist}
\blankline

\textbf{Consultant} at \href{http://www.init.se}{\textbf{Init}}\hfill {November 2013 - June 2016}
\begin{outerlist}
\item[] Consultant for a small consultancy, customers in markets such as broadcasting, telecom and online gambling.

	\item Designed, automated and implemented infrastructure (in code) for Joors. Brought down the provisioning of complete environments to < 1 hour,
		using vSphere, The Foreman and Puppet. The platform was still in use and usable over 5 years after initial implementation.
	\item Custom Puppet module development to manage Gigaspaces, the core of it was later open sourced.
	\item Developed tools to manage and fix subtitles and manifest files for the media archive at Viaplay.
	\item Automated rolling and full restarts of a Hazelcast based application using Saltstack.
\end{outerlist}
\blankline

\textbf{System Administrator} at \href{http://www.avanza.se}{\textbf{Avanza}}\hfill {April 2013 - October 2013}
\begin{outerlist}
\item[] Avanza is a brokerage and savings bank for the Swedish market.

	\item Developed and refactored big parts of our Puppet codebase, DRYing it out and separating data from code.
	\item Coached team members and taught Puppet sensisble practices.
	\item Part of the team that redesigned our data center, including physical and logical network planning and hardware evaluation.
	\item Learned that good leaderships always is more important than cool tech.

\end{outerlist}
\blankline

\textbf{System Administrator} at \href{http://www.ztorm.com}{\textbf{Ztorm}}\hfill {January 2013 - March 2013}
\begin{outerlist}
\item[] System Administrator for a digital media distribution company, doing web and storage operations. 

	\item Leveraged supply\_drop to do push based automation with Puppet and Capistrano.
	\item Introduced the development team to Vagrant and automated their development environments.
\end{outerlist}
\blankline

\textbf{System Administrator} at {\textbf{The Delta Projects}}\hfill {August 2012 - December 2012}
\begin{outerlist}
	\item Part of as small team that designed the new data center, including racking and stacking, physical and logical network design and configuration.
	\item Built Big Data processing clusters based on Hadoop and HDFS.
	\item Fully automated provisioning of machines, using The Foreman, Chef and lots of shell glue.
\end{outerlist}
\blankline

\textbf{Infrastructure Technician} at \href{http://www.unibet.com}{\textbf{Unibet}}  \hfill {March 2011 - July 2012}
\begin{outerlist}
	 \item Designed and implemented the new provisioning and management platform based on The Foreman and Puppet. Planned the release routines, versioning and wrote the majority of our first Puppet modules. Brought down our provisioning times from multiple weeks to minutes.
	\item Introduced and promoted the idea of using Kanban for operations work.
	\item Advocated the idea of using a dedicated third line from technical development to deflect questions regarding infrastructure, network or applications to a dedicated person on a rotating schedule.
\end{outerlist}
\blankline

 \textbf{Systems Specialist, Consultant} at \href{http://www.24solutions.se}{\textbf{24 Solutions}}\hfill {September 2010 - March 2011}
\begin{outerlist}
	\item[] Consulting assignments where I wrote, taught and coached on well written Puppet code, introduced Kanban with WIP limits and task scope breakdown.
\end{outerlist}
\blankline

\textbf{System Administrator} at {\textbf{Squace} \hfill {March 2010 - July 2010}
\begin{outerlist}
\item Squace was a mobile browser built in Java for feature (non smart) phones.
	\item[] Taught myself Puppet and explored MCollective, understood the importance of configuration management as a discipline and infrastructure as code as a practice. Got some experience with MongoDB, PostgreSQL and way to many virtualization platforms (we we're running KVM, VirtualBox and Xen, at the same time).
\end{outerlist}
\blankline

\textbf{System Administration Consultant} at \href{http://www.r2m.se/}{\textbf{R2M}}\hfill {December 2007 - August 2009}
\begin{outerlist}	
\item[] Learned the importance of version control for configuration with RCS and Subversion, racked and stacked machines and learned to plan my network design up front because rearranging network equipment is exhausting, annoying and disrupting. Consulted as application server admin for WebLogic and WebSphere and replaced an aging Windows 2000 AD server with a FLOSS solution based on  OpenLDAP, Samba, Kerberos.
\end{outerlist}
\blankline

\section{Education}
%
\textbf{Linux System Administration} at \href{http://www.jenseneducation.se}{\textbf{JENSEN Education}} \hfill {2005 - 2007}
\blankline
\begin{outerlist}
	\item A mixed bag of courses including (but not limited to) Linux, Sendmail, Bind, Apache and MySQL administration, programming for UNIX in C and Perl, networking in both the practical (switching, routing, TCP/IP, DNS) and theoretical (OSI layers) sense.
\end{outerlist}
\blankline

\section{Additional Engagements}
{Chairperson}, {\textbf{DevOpsdays Stockholm}} \hfill{Since 2018}
\begin{outerlist}
	\item[] Currently serving as chairperson for the Swedish not for profit Devopsdays Stockholm. Founded to help foster an interest in DevOps, mainly through the DevOpsdays Stockholm conference. Part of the global DevOpsdays series of conferences.
\end{outerlist}
\blankline

{Lead Organizer}, \href{https://devopsdays.org/events/2017-stockholm/welcome/}{\textbf{DevOpsdays Stockholm 2017}} \hfill{2017}
\begin{outerlist}
	\item[] DevOpsdays is a global conference organization ran by volunteers, I planned and hosted the very first event in Stockholm, Sweden.
	Tasks included talk selection, sponsor communication and acquirement, marketing and more.
\end{outerlist}
\blankline

{Founder and Organizer}, \href{http://www.meetup.com/DevOps-Stockholm/}{\textbf{DevOps Stockholm}} \hfill{Since 2011}
\begin{outerlist}
	\item[] DevOps Stockholm is a user group for people interested in things like; infrastructure as code, cross functional teams,
		continous delivery and optimizing for sustainable productivity in the organization.
\end{outerlist}
\blankline

{Co-Founder and Organizer}, \href{http://bodyfest.se} {\textbf{Bodyfest}} \hfill{2009 - 2013}
\begin{outerlist}
	\item[] Bodyfest is an annual music festival with a focus on music from or inspired by the 80's alternative electronic music scene. 
	Bands such as Front 242, DAF, Pouppée Fabrikk and The Klinik have performed, approximately 1000 people attend the event.
\end{outerlist}
\blankline

{Organizer and DJ}, {\textbf{Klubb Bodytåget}}, \hfill{2008 - 2013}
\begin{outerlist}
\item[] Klubb Bodytåget (lit. the Bodytrain) was a live club for alternative electronic music rooted in the 80's EBM scene. 
\end{outerlist}

\section{References}
Available upon request.

\section{Languages}
Fluent in Swedish, English and German.
\end{document}

%%%%%%%%%%%%%%%%%%%%%%%%%% End CV Document %%%%%%%%%%%%%%%%%%%%%%%%%%%%%
